%%%%%%%%%%%%%%%%%%%%%%%%%%%%%%%%%%%%%%%%%%%%%%%%%%%
%
%  New template code for TAMU Theses and Dissertations starting Fall 2016.
%
%
%  Original Author: Sean Zachary Roberson
%  This version adapted for URS by Parasol lab.
%  Adapted from version 3.16.10, which was last updated on 9/29/2016.
%  URS adaptation last updated 1/9/2017.
%
%%%%%%%%%%%%%%%%%%%%%%%%%%%%%%%%%%%%%%%%%%%%%%%%%%%
%%%%%%%%%%%%%%%%%%%%%%%%%%%%%%%%%%%%%%%%%%%%%%%%%%%%%%%%%%%%%%%%%%%%%
%%                           ABSTRACT
%%%%%%%%%%%%%%%%%%%%%%%%%%%%%%%%%%%%%%%%%%%%%%%%%%%%%%%%%%%%%%%%%%%%%

\chapter*{ABSTRACT}

\addcontentsline{toc}{chapter}{ABSTRACT} % Needs to be set to part, so the TOC doesnt add 'CHAPTER ' prefix in the TOC.

\pagestyle{plain} % No headers, just page numbers
\pagenumbering{arabic} % Arabic numerals
\setcounter{page}{1}
\begin{center}

\begin{singlespace}
\abstracttitle
\end{singlespace}
\vspace{2em}
\begin{singlespace}
\author \\
Department of \department \\
Texas A\&M University \\
\end{singlespace}
\vspace{2em}
\begin{singlespace}
Research Advisor: Dr. \ursadvisor \\
Department of \advisordepartment \\
Texas A\&M University \\
\end{singlespace}
\end{center}
\vspace{2em}

% \indent Recommending compatible items to users is an increasingly important research topic, and recommender systems are used extensively in different applications varying across domains to recommend items from books to music. e-Commerce systems such as Amazon and Netflix depend on recommender systems to increase their profits by recommending products the consumers are interested in against other products.

% \indent Current recommender systems recommend items based on two factors: user and items. For example, if a user buys a certain product, then the recommendation system will recommend similar products or products you have already purchased. For certain categories, the focus of compatibility relationship between products should be analyzed and used to recommend products to offer a complementary product, not just a similar product.

% \indent Our research proposes that compatibility can provide more accurate recommendations versus traditional recommender systems. This is especially true for electronics, so we will focus our research on electronics initially, and given time, we will progress to other categories. Through compatibility recommender systems, we will define compatibility for electronics, create a model to identify compatible products in electronics, analyze large product datasets and their relationships, and create a method to provide analytics for our results with recommender systems. Furthermore, our research differs from current market recommendation systems in that we will propose a recommendation system focused on compatibility and efficiency of the systems to provide user results.

% \indent Our overall challenge after we have created a new definition of compatibility is to analyze features that give us information about compatibility. This information includes text information, such as descriptions and product names, and image data. Another challenge is we must consider specific distinguishing features such as brand names. Next, we will find relationships between features that will give an appropriate compatible recommendation from our existing definition compared with existing recommendation that utilizes similar substitution methods. 


\indent Item recommendation is an increasingly important research topic that focuses on analyzing the relationships between products to recommend items to users based on their preferences or previous activity. These systems are used extensively in different applications varying across domains to recommend items ranging from books to music. Many companies, such as Amazon, Netflix, and Spotify, leverage recommender systems to drive further engagement and revenue by delivering value through a scalable way of personalizing content for their users.

\indent Current recommender systems recommend items based on two factors: users and items. For example, if a user purchases a product, then the recommender system will recommend similar products based on the users' previous purchases or similar social circles. In certain domains, such as clothing and electronics, the focus of compatibility relationships between products should be analyzed and used to recommend products to offer a complementary product, not a similar one.

\indent In our thesis, we propose a new definition of compatibility to provide a new and improved recommender system strictly for item compatibility. Compared to traditional recommender systems, compatibility recommender systems provide more accurate item recommendations for users. Our thesis currently focuses on analyzing the compatibility relationships within top-level categories in Amazon data but can be applied to any domain where compatibility is important. In order to do so, we define a general definition of compatibility, analyze a large product dataset and map product relationships, create a model to identify compatible items, and compare our results with other models. We will be analyzing the Cell Phone & Accessories category with our compatibility definition. Compared to other recommender systems, our compatibility recommender system is able to recommend compatible items at a higher accuracy and can therefore be used to provide users with a more personalized experience.

\pagebreak{}
